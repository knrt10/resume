%----------------------------------------------------------------------------------------
%	PACKAGES AND OTHER DOCUMENT CONFIGURATIONS
%----------------------------------------------------------------------------------------
% !TeX spellcheck = en
\documentclass{resume} % Use the custom resume.cls style
\usepackage{hyperref}
\hypersetup{
    colorlinks=true,
    linkcolor=cyan,      
    urlcolor=cyan,
}
 
\urlstyle{same}

\usepackage[left=0.75in,top=0.52in,right=0.75in,bottom=0.52in]{geometry} % Document margins
\usepackage[colorlinks=true, linkcolor=black, urlcolor=black]{hyperref}
\usepackage{marvosym}
\usepackage{graphicx}

\name{Kautilya Tripathi}
\address{\href{https://knrt10.github.io/}{knrt10.github.io} \\
         \href{mailto:tripathi.kautilya@gmail.com}{tripathi.kautilya@gmail.com}\ \\
         \href{https://www.linkedin.com/in/knrt10/}{in/knrt10} \\
         \href{https://github.com/knrt10/}{github.com/knrt10}}

\begin{document}

%----------------------------------------------------------------------------------------
%	SUMMARY SECTION
%----------------------------------------------------------------------------------------

\begin{rSection}{Summary}
  \begin{rSummarySection}
  {
    \item Experience with backend web services and open source development. Interested in SRE, distributed systems, dev tools and systems programming.
    \item Go, JavaScript, TypeScript, Kubernetes, ElasticSearch, MongoDB, git
  }
  \end{rSummarySection}
\end{rSection}

%----------------------------------------------------------------------------------------
% WORK SECTION
%----------------------------------------------------------------------------------------

\begin{rSection}{Experience}
  \begin{rWorkSection}{Open Mainframe Project}
                           {Remote}
                           {Open Mainframe Intern}
                           {June -- Sep '19}
  {
    \item Built docker images fors 390x architecture on SUSE along with a bot to keep the images in sync with Dockerhub and Github.   
    \item Setup Jenkins CI pipeline. \href{https://github.com/openmainframeproject-internship/DockerHub-Development-Stacks/}{Project's Repository} and \href{https://github.com/knrt10/docker-hub-development-stacks-bot/}{Bot Repository}.  
  }                         
  \end{rWorkSection}

  \begin{rWorkSection}{Rocket.Chat}
                      {Remote}
                      {Google Summer of Code Student}
                      {May -- Sep '19}
  {
    \item Added new feature of Realtime-Monitoring to their product Livechat. \href{http://bit.ly/2kGqWt2}{GSoC Work Report}.
  }
  \end{rWorkSection}

  \begin{rWorkSection}{Atlan}
                     {New Delhi, IN}
                     {Software Development Intern (Backend)}
                     {Jan -- July '19}
  {
    \item Worked on Baseline Upload for their software collect. 
    \item Rewrote Collect's File Upload which increased the speed and error handling by 99%. Used Jmeter and Blazemeter for load testing. Deployed their code to development and staging.
    \item Implemented task queues and workers using SQS and RabbitMQ.
    \item Rewrote their module response validation in Golang.
  }
  \end{rWorkSection}

  \begin{rWorkSection}{Eulercoder}
                     {Remote}
                     {Contract Developer}
                     {1 week '18}
  {
    \item Developed admin API where one can see all users and manage them.
    \item Worked on frontend for admin in React.
  }
  \end{rWorkSection}
\end{rSection}

%----------------------------------------------------------------------------------------
%	PROJECTS SECTION
%----------------------------------------------------------------------------------------

\begin{rSection}{Projects}
  \begin{rProjectSection}
    \item \textbf {Typeracer-cli} -- Learn how to touch type from terminal. Written in \href{https://github.com/p-society/typeracer-cli/}{TypeScript}.
    \item \textbf {asciiConvert} -- Get ascii art in terminal. Written in \href{https://github.com/knrt10/asciiConvert/}{Golang}
    \item \textbf {cloudinary-cli} -- CLI tool for managing files in \href{https://cloudinary.com/}{cloudinary.com}. Code can be \href{https://github.com/knrt10/cloudinary-cli/}{found here}
    \item My other projects can be \href{https://knrt10.github.io/projects/}{found here}
  \end{rProjectSection}

  \begin{rBlurbSection}
    \item {\em Open-source contributions:}
      I am an active OSS contributor. My \href{http://bit.ly/2kdr9Ui}{\textbf{Merged PR's}} and \href{http://bit.ly/2kxjvV9}{\textbf{open PR's}}.
  \end{rBlurbSection}
\end{rSection}

%----------------------------------------------------------------------------------------
%	VOLUNTEER EXPERIENCE
%----------------------------------------------------------------------------------------

\begin{rSection}{Volunteer Experience and Honours}
  \begin{tabular}{rll}
2018	     & {\href{https://github.com/p-society/}{P-Society}}  & Led the college programming club.\\
2018	     & FreeCodeCamp  & Got \href{https://www.freecodecamp.org/knrt10/}{4 certificates}.\\
\end{tabular}
\end{rSection}

%----------------------------------------------------------------------------------------
%	EDUCATION SECTION
%----------------------------------------------------------------------------------------

\begin{rSection}{Education}
  \begin{rEducationSection}{International Institute of Information Technology, Bhubaneswar}
                           {Aug '16 -- Present}
                           {Bachelor of Technology in Computer Science and Engineering}
  \end{rEducationSection}
\end{rSection}

\end{document}
